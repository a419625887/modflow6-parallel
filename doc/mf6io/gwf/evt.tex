Input to the Evapotranspiration (EVT) Package is read from the file that has type ``EVT6'' in the Name File. Any number of EVT Packages can be specified for a single groundwater flow model. All single-valued variables are free format.

Evapotranspiration input can be specified using lists or arrays, unless the DISU Package is used.  List-based input must be used if discretization is specified using the DISU Package.  List-based input for evapotranspiration is the default, and is described here.  Instructions for specifying array-based evapotranspiration are described in the next section. 

List-based input offers several advantages over the array-based input for specifying evapotranspiration.  First, multiple list entries can be specified for a single cell.  This makes it possible to divide a cell into multiple areas, and assign a different evapotranspiration rate or extinction depth for each area (perhaps based on vegetation type or some other criteria).  In this case, the user would likely specify an auxiliary variable to serve as a multiplier.  This multiplier would be calculated by the user and provided in the input file as the fractional cell are for the individual evapotranspiration entries.  Another advantage to using list-based input for specifying evapotranspiration is that boundnames can be specified.  Boundnames work with the Observations capability and can be used to sum evapotranspiration rates for entries with the same boundname.  A disadvantage of the list-based input is that one cannot easily assign evapotranspiration to the entire model without specifying a list of model cells.  For this reason \mf also supports array-based input for evapotranspiration.

ET input is read in list form, as shown in the PERIOD block below. Each line in the PERIOD block defines all input for one cell. Entries following \texttt{cellid}, in order, define the ET surface (\texttt{etss}), maximum ET flux rate (\texttt{etsr}), extinction depth (\texttt{etsx}), all (\texttt{netseg} -- 1) \texttt{pxdp} values, all (\texttt{netseg} -- 1) \texttt{petm} values, all auxiliary variables (if AUXILIARY option is specified), and boundary name (if BOUNDNAMES option is specified).

\vspace{5mm}
\subsubsection{Structure of Blocks}
\vspace{5mm}

\noindent \textit{FOR EACH SIMULATION}
\lstinputlisting[style=blockdefinition]{./mf6ivar/tex/gwf-evt-options.dat}
\lstinputlisting[style=blockdefinition]{./mf6ivar/tex/gwf-evt-dimensions.dat}
\vspace{5mm}
\noindent \textit{FOR ANY STRESS PERIOD}
\lstinputlisting[style=blockdefinition]{./mf6ivar/tex/gwf-evt-period.dat}
\packageperioddescription

\vspace{5mm}
\subsubsection{Explanation of Variables}
\begin{description}
\input{./mf6ivar/tex/gwf-evt-desc.tex}
\end{description}

\vspace{5mm}
\subsubsection{Example Input File}
\lstinputlisting[style=inputfile]{./mf6ivar/examples/gwf-evt-example.dat}

\vspace{5mm}
\subsubsection{Available observation types}
Evapotranspiration Package observations are limited to the simulated evapotranspiration flow rates (\texttt{evt}). The data required for the EVT Package observation type is defined in table~\ref{table:gwf-evtobstype}. Negative and positive values for an observation represent a loss from and gain to the GWF model, respectively.

\begin{longtable}{p{2cm} p{2.75cm} p{2cm} p{1.25cm} p{7cm}}
\caption{Available EVT Package observation types} \tabularnewline

\hline
\hline
\textbf{Stress Package} & \textbf{Observation type} & \textbf{ID} & \textbf{ID2} & \textbf{Description} \\
\hline
\endhead

\hline
\endfoot

\input{../Common/gwf-evtobs.tex}
\label{table:gwf-evtobstype}
\end{longtable}

\vspace{5mm}
\subsubsection{Example Observation Input File}
\lstinputlisting[style=inputfile]{./mf6ivar/examples/gwf-evt-example-obs.dat}
